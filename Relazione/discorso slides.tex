In questa esperienza di tirocinio si vogliono definire le condizioni ottimali per eseguire delle misure reologiche su fluidi non newtoniani utilizzando un apparato ad hoc invece che un reometro. Per poterlo fare ci concentriamo su risultati noti, ovvero sui fluidi Newtoniani, per i quali sappiamo che cosa ci aspettiamo.

Introduciamo un po' di teoria: La mircrofluidica è il settore scientifico che si occupa dello studio e della manipolazione di piccoli volumi di liquidi ($\mu L/pL$), confinati su scale submillimetriche. In particolare, ciò di cui ci occuperemo è microfluidica aperta, ovvero un caso particolare in cui si hanno piccoli volumi discreti depositabili su superfici aperte con interfaccia libera liquido/aria.
In tali regimi domina il contributo capillare rispetto a quello gravitazionale, per cui il numero di Bond, definito come il rapporto tra le due, è molto basso, dell'ordine dei 10 alla -12.

Le forze capillari provengono dal concetto di tensione spuerficiale che rappresenta l'energia fornita dall'esterno per ottenere un aumento unitario della superficie dell'interfaccia: (leggo la formula). Ciò significa che per aumentare la superficie serve lavoro, e che la tensione superficiale è la forza capillare per unità di lunghezza.
Passiamo ora a delle definizioni di tipo geometrico che caratterizzano la goccia:
L'Angolo di contatto è definito dalla legge di Young qui riportata, che lega la tensione superficiale liquido/aria appena introdotta con le tensioni solido/gas e solido/liquido. dipende dalla goccia ma anche dal substrato.

In base a come reagiscono fra di loro infatti può essere definito il concetto di bagnabilità, dipendente strettamente dall'angolo di contatto.
leggo i valori, e spiego che nel nostro caso i modelli utilizzati ci consentono optare per substrati idrofobici.

la linea di contatto è la linea che separa le tre fasi. Quindi è sostanzialmente questa linea (indicandola con il cursore).L' isteresi è quando l'angolo di contatto varia per la presenza di asperità, imperfezioni o disomogeneità fisiche o chimiche, in un intervallo $\theta=[\theta_R;\theta_A]$ (rispettivamente angolo di Reciding e Advancing). Mentre il pinning è quando quando la linea di contatto rimane bloccata e l'angolo di contatto subisce un fenomeno di isteresi. Quindi l'isteresi è direttamente collegata al pinning: se si ha pinning si ha isteresi dell'angolo, in quanto la goccia se sollecitata in qualche modo, per la proprietà di incomprimibilità del volume tenderà ad oscillare, e così oscillerà anche l'angolo se la goccia è pinnata.

Introduciamo ora il concetto di viscosità, secondo parametro che andremo a caratterizzare per ogni fluido. La viscosità descrive la resistenza di un materiale allo scorrimento.  Per poter introdurre il suo significato fisico introduciamo lo shear rate e lo shear stress, rispettivamente la variazione di velocità del fluido nell'unità di spazio in direzione della sezione in cui si trova il fluido, e lo sforzo di taglio, la forza esterna applicata per unità di spazio. la costante di proporzionalità fra le due è la viscosità, e determina il comportamento del fluido: se costanto, è newtoniano (ad esempio acqua, miele, glicerolo) se dipende dalla tenzione superficiale sono non newtoniani, in particolare in maniera direttamente proporzionale abbiamo fluidi che tendono ad assumere comportamenti solidi, irrigidirsi, e abbiamo fluidi shear thickening, al contrario se manifestano comportamenti ancora più liquidi, diminuendo la viscosità, si parla di fluidi shear thinning.

Passiamo ai modelli usati. Nei modelli proposti la goccia è modellizzata attraverso un oscillatore armonico smorzato. la componente elastica è data dalla combinazione di tensione superficiale e incompribilità del volume, e varia in base al tipo di fluido: se è non newtoniano abbiamo ifatti un ulteriore componente, dovuta all'elasticità di bulk. La dissipazione invece è dovuta alla viscosità, che appunto attenua le oscillazioni fino a fermarle.
Le altre ipotesi sono: La linea di contatto è pinnata, il substrato è idrofobico, le forze capillari predominano sulla forza di gravità. Il che rende le goccie molto sferiche. quest'ultima ipotesi è anche chiamata "regime di capillarità" che coinvolge spesso volumi bassi di goccie.

Passiamo ai modelli usati:
Sharp fece per primo uno studio utilizzando l'apparato che riproporrò, la sua è un analisi dei modi verticali per gocce in regime di capillarità in modo da poter considerare la superficie come un arco di circonferenza. Si riscontra un'omogenea sovrastima di un fattore sperimentale di $\alpha\simeq0.81$.
Temperton migliora la formula di Sharp trovando tramite considerazioni geometriche il fattore h, ovvero l'altezza della goccia.

passiamo al calcolo della viscosità. dalla tesi di Temperton si ricavano due contributi del parametro gamma. Uno, che in genere è sufficiente, è quello del contributo di bulk, che coinvolge la geometria della goccia, vediamo infatti comparire la lunghezza L del profilo della goccia. Il secondo invece è da introdurre soprattutto nel caso di idrofilicità o poca idrofobicità, ed è un contributo in cui compare il fattore di correzione di Sharp.

Lo scopo dell'esperienza è quello di riprodurre un reometro, uno strumento molto costoso in grado di poter imprimere degli sforzi di taglio in maniere molto precise e particolari a grandi volumi di fluidi al fine di studiarne le proprietà reologiche. La stessa funzione può essere riprodotta utilizzando un apparato comprensivo di laser che colpisce una goccia, sollecitata da un piccolo puff di gas, il laser viene deflesso e catturato in modo da poterne calcolare la trasformata di fourier e ricavare gli spettri di risonanza, dai quali si ottengono frequenze di risonanza e FWHM per calcolare tensione superficiale e viscosità. Il procedimento viene eseguito per i liquidi newtoniani, dei quali conosciamo già le proprietà. una volta confermato o smetito il funzionamento dell'apparato si può passare allo studio di liquidi viscoelastici.

come studio preliminare della situazione ci siamo dovuti ricavare parametri geometrici della goccia, ed in base ad essi ottimizzare l'esperimento in sè per sè. il primo setup consiste in un led focalizzato, un telecamera allied per catturare video e immagini, una pompa siringa per controllare il flusso del fluido in uscita, due substrati differenti per scegliere quello ottimale (teflon/parafilm) un programma labview per aquisire video. con tale setup si è caratterizzato l'angolo di contatto, la sua isteresi(che fosse abbastaza ampia da permettere le oscillazioni) e si è eseguito uno studio della tensione superficiale preliminare con un metodo notoriamente funzionante.

dall'analisi con labview si può vedere che catturando le immagini e caricandole si seleziona la porzione in cui eseguire misure di angoli di contatto, aggiustando la foto nella luminosità e contrasto. Si evince così che per mantenere le ipotesi di idrofobicità è ottimale l'uso del teflon, e con esso oltretutto c'è un più ampio fenomeno di isteresi.
Nella tabella sono riportati i risultati di angoli di cotatto differenti.

per lo studio dell'isteresi abbiamo dovuto portare il sistema agli estremi. Utilizzando la pompa e con il sistema di acquisizione video che ritroviamo nella slide precedente abbiamo potuto generare dei frame ogni frazione di secondo che sembrava ottimale, in modo da poter catturare una buona quantità di immagini che vediamo qui: nella prima la goccia si sta riempiendo di fluido, e l'angolo cresce perchè la linea di contatto è pinnata. quando il volume aumenta abbastanza da depinnare la goccia questa avanza, mantenendo l'angolo più ampio possibile, che sarà quindi theta advancing. sotto è descritto il processo inverso, quando viene ritirato il fluido e quindi diminuito l'angolo.

Si è utilizzato imageji invece per calcolare il profilo della goccia L: con un tool è possibile estrarre il raggio della circonferenza tracciata, e sapendo che L=2tethaR siamo apposto. vi è poi un tool molto utile al fine di utilizzare il metodo della pendant drop. Il tool ricava direttamente il valore della viscosità, inserendo il valore della densità del fluido e la scala in pixels.

Finite le misure preliminari possiamo passare al vero setup: nella ricostruzione possiamo notare che un laser attraversa la goccia di fluido e colpisce un fotodiodo. Un piccolo puff prodotto semplicemente soffiando sulla goccia genera le oscillazioni nella goccia, che deflette il raggio laser. Tale deflessione può essere catturata tramite un programma labview che utilizza una scheda di acquisizione dati che mostra il damping dell'oscillazione del segnale di voltaggio. 

questo è il segnale di voltaggio riprodotto e si nota molto chiaramente come per viscosità maggiori, ossia il segnale a destra che mostra una soluzione di glicerolo all'85\% pura, si ha una dissipazione più immediata, maggiore.

Il movimento che viene prodotto dalle goccie si basa su questo "soffio" che imprimevo alle goccie, nel tentativo di essere più precisa possibile nella direzione e l'intensità tutte uguali non sempre riuscivo, spesso fallivo: nei seguenti video si può notare come nel primo caso, per un volume minore di goccia è più semplice far avvenire la vibrazione, nel caso di volumi maggiori è più difficile dare un impulso abbastanza grande sa far avvenire l'oscillazione ma non così grande da depinnare la goccia. Il sistema video è stato anche utile al fine di controllare che il volume della goccia non diminuisse per fenomeni di evaporazione ed è stato utile per ricostrurire eentuali anomalie nei segnali, come ad esempio in questo caso.

Per analizzare i picchi è stata eseguita la trasformata di fourier del segnale ed è stato utilizzato il tool peak analizer di Origin.

I risultati ottenuti per acqua ultrapura sono incoraggianti per il primo modo normale di vibrazione (ossia n=2) ma non per i seguenti, che sembrano variare di molto per entrambi i modelli. Tutt'ora non sono sicura di che picchi io abbia visto, può essere che si trattava di picchi dovuti a luce ambientale o simili, anche se comunque per fare le misure ho spento la luce. in ogni caso si è dunque scelto di utilizzare solo il primo modo normale di vibrazione per il calcolo delle proprietà reologiche, e di utilizzare come confronto il modello di temperton, perchè più preciso in fattori geometrici e presenta un errore molto più basso. L'errore sperimentale è stato trovato utilizzando la deviazione standard, l'errore teorico con il metodo di propagazione degli errori.

Nei grafici notiamo quanto i dati sperimentali fossero fitti nella trasformata di fourier, la maggior quantità di dati trovati è stata per la coppia numero di campioni pari a 8000 con rate di campionamento 1000Hz.
Nell'ultima slide si può notare come molto spesso io abbia dovuto rifare le misure notando che ci fosse qulcosa che non va negli spettri. L'errore che si riscontra se la goccia si depinna è uno sdoppiamento del picco, questo perchè la geometria della goccia tende a cambiare per via dell'allungamento nella direzione de soffio, come abbiamo visto dal video, le due direzioni splittano il picco che risulta degenere se la geometria è simmetrica, evidenziando un differente modo di vibrazione per direzione.


i grafici mostrati sono per l'acqua UP, i cui il primo era per far notare la differenza sostanziale fra il caso con fattore di correzione e quello no. Inoltre si nota quanto siano vicini i dati sperimentali ai due modelli per il primo modo di vibrazione.
Sotto si possono notare dei grafici f^2 versus 1/m, e ci aspettiamo una relazione lineare fra le due, guardando alla formula di Temperton. Vi sono due grafici differenti perchè si hanno due set di dati differenti: inizialmente si erano presi dei volumi nel range di volumi scelto (20muL-100muL) ma poi si è voluto estendere il problema a volumi più bassi, cercando i modi normali di vibrazione più alti, senza successo. comunque il primo modo normale risulta sempre in accordo.

I risultati per il glicerolo sono di nuovo buoni, in quanto si nota un buon accordo, a parte per un valore, quello di 20muL di glicerolo 85\%, probabilmente dovuto ad una sovrastima della quantità di volume.

di nuovo questi sono gli spettri ricavati per il caso di glicerolo+acqua, molto spesso si riusciva a visualizzare solo il primo modo di vibrazione. si può notare anche come al crescere della quantita di glicerolo in acqua( e quindi della viscosità) come si allargano i picchi, infatti la FWHM rappresenta la gamma.

I risultati hanno tutti un buon accordo all'interno dell'errore sperimentale, a parte per quel valore di cui parlavamo prima che si può ragionevolmente scartare. accanto si ha il grafico scartando il punto.

Passiamo dunque ai risultati: per quanto riguarda la tensione superficiale, la formula utilizzata è quella ricavata da Temperton, sono prese in considerazione solo i valori consistenti, e dunque solo i valori della frequenza n=2.
Di seguito sono riportati i valori teorici, quelli sperimentali ricavati con il metodo pendent drop, quelli sperimentali riavati con l'apparato. Si nota che tutti i valori rientrano nell'errore sperimentale, nei valori teorici. L'errore così grande è dovuto alla grande variabilità dei risultati, causata con grande probabilità dalla statistica bassa (5 volumi per ognuno) e la variabilità del soffio.

Per quanto riguarda la viscosità, più aumenta la viscosità, più errore c'è. Questo perchè si è notato che più aumenta la viscosità più risulta difficile ottenere gli spettri, perchè è più difficile controllare il soffio. Inoltre la statistica è veramente molto povera, infatti il primo valore del glicerolo all'85\% è anche da scartare, mentre quello da 100muL è una sola misura. Analogamente per il gly75\%, abbiamo solo 3 misure da prendere in considerazione, e così anche il gly 60\% , è stato impossibile ricavare i valori più alti di volume. Come si può notare dalle figure precedenti infatti si nota solo un grande picco a valori successivi, sia per 60 che per 75, come se il modo n=2 con il solo soffio sia complicato da poter eccitare raggiunta una certa viscosità.

Graficamente ho deciso di plottare in scala logaritmica la FWHM e L^-2 perchè ci aspettiamo abbiano un andamento lineare. dunque per quanto riguarda l'acqua siamo nei parametri, abbiamo una buona statistica e infatti torna. Nel caso del glicerolo la staistica è molto povera ed è difficile poterne trarre un qualche significato fisico utile da questo grafico. Come dicevo il problema è dovuto alla bassa statistica, in più avevo una quantità grande di picchi da scartare, infatti come possiamo vedere dall figure, i picchi spesso erano frastagliati, evidenziando l'esistenza di tanti altri picchi più piccoli, che il peak analizer mi trovata ma sbagliando la FWHM. dunque il fatto che il valore non sia centrato bene può anche essere dovuto alla sottostima del peak analizer delle fwhm.

in conclusione ciò che si può dire è che i problemi riscotrati sono soffio non controllato, Errori troppo grandi, Forma dei picchi non sempre ottimale, Ad ognuno posso attribuire una possibile soluzione: per il soffio consiglio di procurarsi  un diffusore di gas cotrollabile a distanza, in modo da poterne controllare l'intensità e scegliere tipologia di gas (tipo Azoto). poi per l'errore grande ovviamente si deve Aumentare la statistica e migliorare le condizioni ambientali. Di fatto c'è anche da dire che i valori potevano essere poco corretti perchè c'è molta variabilità per quanto riguarda la temperatura della stanza. alle volte le misure sono state fatte a condizionatore acceso, altre spento, e in piena estate vi sono anche 10 gradi di differenza. Poi per aumentare l'idrofobicità si può pensare a bagnare il substrato con sostanze oleose, in modo da  rendere la goccia pressochè sferica. Per la forma dei picchi le cose sono due: si migliora la qualità del puff di gas e anche si elimina il rumore di fondo, registrando il segnale senza soffio e rimuovendolo da quello con il soffio, o si cambia analizzatore di picchi, usando ad esempio un software customizzato(si usa root, python..).
 








